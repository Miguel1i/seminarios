\documentclass[a4paper,12pt]{article}
\usepackage[utf8]{inputenc}      % Codificação de caracteres
\usepackage[T1]{fontenc}         % Codificação de fontes
\usepackage[portuguese]{babel}   % Português de Portugal
\usepackage{hyperref}            % Hiperligações
\usepackage{geometry}            % Definição de margens
\usepackage{graphicx}
\usepackage{float}
\usepackage[acronym]{glossaries} % Definição do glossário
\usepackage{indentfirst}

\geometry{a4paper, margin=2.5cm}


\makeglossaries


\newglossaryentry{Tempora}{
    name=Tempora,
    description={Programa de vigilância em massa operado pelo GCHQ do Reino Unido.  Interceptava e armazenava grandes quantidades de dados de comunicações por cabos de fibra ótica}
}

\newglossaryentry{PRISM}{
    name=PRISM,
    description={Programa de vigilância da NSA que coletava dados diretamente de empresas como Google, Facebook e Microsoft, permitindo monitorizar comunicações eletrônicas de indivíduos ao redor do mundo}
}

\newglossaryentry{XKeyscore}{
    name=XKeyscore,
    description={Ferramenta de análise da NSA usada para pesquisar e analisar dados de internet em tempo real, permitindo a monitorização de atividades online de qualquer usuário sem necessidade de autorização prévia}
}

\newacronym{nsa}{NSA}{National Security Agency / Agência de Segurança Nacional (EUA)}
\newacronym{cia}{CIA}{Central Intelligence Agency / Agência Central de Inteligência (EUA)}

\title{
    \vspace{-1cm} 
    \begin{figure}[H]
        \centering
        \includegraphics[width=0.3\textwidth]{./images/uac_logo.png}
    \end{figure}
    \vspace{0.5cm} 
    \textbf{Seminários}
    \\
    Ética
}

\author{Henrique Pacheco, Manuel Estrela, Miguel Rego}
\date{Docente: Maria Neves \\ \vspace{0.5cm}Licenciatura em Informática }

\begin{document}

\maketitle

\begin{abstract}
% Conteúdo do resumo
\end{abstract}

\newpage
\tableofcontents

\newpage

\section{Apresentação do Caso}

Em maio de 2013, o ex-funcionário da Agência de Segurança Nacional (NSA) dos Estados Unidos, Edward Snowden, divulgou material sensível ao \textit{The Guardian} e ao \textit{Washington Post} expondo a existência de vários programas de vigilância em massa - tais como o \gls{Tempora}, \gls{PRISM} e \gls{XKeyscore} - que permitiam a recolha e monitorização de comunicações de milhões de pessoas, não só em território norte-americano, como também em vários países.

O que desencadeou um debate global sobre os limites do poder estatal, a proteção dos direitos à privacidade e a necessidade de transparência e controlo nos procedimentos de segurança.
Tendo em conta o que nos foi ensinado em um módulo de ética, o caso não só revela falhas jurídicas, como também éticas, uma vez que a ação do Snowden coloca em causa a razão e a justificação da ação humana.

Disto isto, dependendo do ponto de vista de cada um, as ações de Snowden foram consideradas ou um ato horrível ou um ato corajoso. Os defensores do mesmo vêem-no como um “herói” e um “patriota” devido ao facto do Governo dos Estados Unidos, através da \acrshort{nsa}, ter ido longe demais nas suas operações de monitoramento dos seus cidadãos e dos governantes de países aliados.
\cite{tavani} 

\section{Factos}

Edward Snowden, que acumulou experiência na \acrshort{cia} e na \acrshort{nsa}, teve acesso a um vasto conjunto de documentos sigilosos enquanto prestava serviços para empresas contratadas pelo governo dos EUA. Insatisfeito com a recolha indiscriminada de dados – prática que viola o direito à privacidade –, decidiu agir para expor estes procedimentos. Em maio de 2013, deslocou-se para Hong Kong, onde entregou parte dos documentos a jornalistas de renome, como os do \textit{The Guardian} e do \textit{Washington Post}. Os documentos revelados demonstraram que a NSA implementava sistemas de interceptação que recolhiam registos de chamadas, e-mails, mensagens instantâneas e outras atividades digitais, sem o consentimento dos cidadãos. Como consequência, o governo norte-americano acusou-o de espionagem, furto de propriedade do Estado e divulgação indevida de informações classificadas, cancelando o seu passaporte e forçando-o a procurar asilo – primeiro numa zona de trânsito no Aeroporto de Moscovo, e depois na Rússia.




\section{Partes Interessadas}

\section{Princípios Violados}

\section{Resolução do Problema}

\section{Considerações Finais}

\begin{thebibliography}{10}

    \bibitem{tavani}
        Tavani, H. T., \& Grodzinsky, F. S. (2014). Trust, betrayal, and whistle-blowing: Reflections on the Edward Snowden case. ACM SIGCAS Computers and Society, 44(3), 8-13.
    \bibitem{teresa}
    Teresa, M. (2024). Revelações de Edward Snowden: Legitimidade da acusação de espionagem à luz do capitalismo da vigilância e da desobediência civil. Handle.net. \url{http://hdl.handle.net/10362/169656}.

    
    
    \bibitem{chatgpt}
        OpenAI. (2024). \textit{ChatGPT}. OpenAI. Disponível em: \url{https://chatgpt.com/}. Versão 4o mini.
    
   
\end{thebibliography}

\printglossaries

\end{document}
