\documentclass[a4paper,12pt]{article}
\usepackage[utf8]{inputenc}      % Codificação de caracteres
\usepackage[T1]{fontenc}         % Codificação de fontes
\usepackage[portuguese]{babel}   % Português de Portugal
\usepackage{hyperref}            % Hiperligações
\usepackage{geometry}            % Definição de margens
\usepackage{graphicx}
\usepackage{float}
\usepackage[acronym]{glossaries} % Definição do glossário
\usepackage{indentfirst}

\geometry{a4paper, margin=2.5cm}


\makeglossaries

% Definição de acrónimos
\newacronym{nsa}{NSA}{National Security Agency}
\newacronym{cia}{CIA}{Central Intelligence Agency}
\newacronym{gchq}{GCHQ}{Government Communications Headquarters}

% Definição de termos normais no glossário
\newglossaryentry{tempora}{
    name=Tempora,
    description={Programa de vigilância em massa operado pelo GCHQ do Reino Unido. Interceptava e armazenava grandes quantidades de dados de comunicações por cabos de fibra ótica}
}

\newglossaryentry{prism}{
    name=PRISM,
    description={Programa de vigilância da NSA que coletava dados diretamente de empresas como Google, Facebook e Microsoft, permitindo monitorizar comunicações eletrônicas de indivíduos ao redor do mundo}
}

\newglossaryentry{xkeyscore}{
    name=XKeyscore,
    description={Ferramenta de análise da NSA usada para pesquisar e analisar dados de internet em tempo real, permitindo a monitorização de atividades online de qualquer usuário sem necessidade de autorização prévia}
}


\title{
    \vspace{-1cm} 
    \begin{figure}[H]
        \centering
        \includegraphics[width=0.3\textwidth]{./images/uac_logo.png}
    \end{figure}
    \vspace{0.5cm} 
    \textbf{Seminários}
    \\
    Ética
}

\author{Henrique Pacheco, Manuel Estrela, Miguel Rego}
\date{Docente: Maria Neves \\ \vspace{0.5cm}Licenciatura em Informática }

\begin{document}

\maketitle

\begin{abstract}
% Conteúdo do resumo
\end{abstract}

\newpage
\tableofcontents

\newpage

\section{Apresentação do Caso}

Em maio de 2013, Edward Snowden, ex-analista da NSA, divulgou dados sigilosos ao The Guardian e ao Washington Post, expondo a existência de vários programas de vigilância em massa, como o \gls{tempora}, \gls{prism} e \gls{xkeyscore}, que permitiam a coleta e supervisão das comunicações de milhões de pessoas, tanto nos Estados Unidos quanto em outros países.

O que fomentou um debate global sobre as limitações do poder estatal, a proteção dos direitos de privacidade e a necessidade de transparência e supervisão em processos de segurança \cite{teresa}.
Tendo em conta o que nos foi ensinado em um módulo de ética, o caso não só revela falhas jurídicas, como também éticas, uma vez que a ação do Snowden coloca em causa a razão e a justificação da ação humana. 

Disto isto, dependendo do ponto de vista de cada um, as ações de Snowden foram consideradas ou um ato horrível ou um ato corajoso. Os defensores do mesmo vêem-no como um “herói” e um “patriota” devido ao facto do Governo dos Estados Unidos, através da \acrshort{nsa}, ter ido longe demais nas suas operações de monitoramento dos seus cidadãos e dos governantes de países aliados \cite{tavani}. 

\section{Factos}


Ao longo da carreira de Edward Snowden, este foi acumulando experiência na \acrshort{cia} e na \acrshort{nsa}, no início de 2013, o mesmo ocupou um posto na Bozz Allen Hamilton, empresa com ligações à Agência norte-americana no Havai.

Durante o seu tempo no Havai, este obteve acesso a um vasto conjunto de documentos sigilosos que violavam o direito à privacidade, Edward Snowden, pediu ao seu supervisor algumas semanas de férias e deslocou-se para Hong Kong onde onde entregou parte dos documentos a jornalistas de renome, como os do \textit{The Guardian} e do \textit{Washington Post}.

Os documentos em questão relatavam o facto de que a \acrshort{nsa} implementava sistemas de interceptação que recolhiam registos de chamadas, e-mails, mensagens e outras atividades digitais, sem o conhecimento dos cidadãos. Como consequência o governo norte-americano acusou-o de espionagem, roubo de propriedade do Estado e divulgação inadequada de informações classificadas, cancelando o seu passaporte e coagindo-o a procurar asilo - primeiro numa zona de trânsito no Aeroporto de Moscovo, e depois na Rússia \cite{pilati}.


\section{Partes Interessadas}

\section{Princípios Violados}

\section{Resolução do Problema}

\section{Considerações Finais}

\begin{thebibliography}{10}

    \bibitem{teresa}
    Teresa, M. (2024). Revelações de Edward Snowden: Legitimidade da acusação de espionagem à luz do capitalismo da vigilância e da desobediência civil. Handle.net. \url{http://hdl.handle.net/10362/169656}.


    \bibitem{tavani}
        Tavani, H. T., \& Grodzinsky, F. S. (2014). Trust, betrayal, and whistle-blowing: Reflections on the Edward Snowden case. ACM SIGCAS Computers and Society, 44(3), 8-13.
    
    \bibitem{pilati}
    Pilati, J. I., \& Olivo, M. V. C. D. (2014). Um novo olhar sobre o direito à privacidade: caso Snowden e pós-modernidade jurídica. Sequência (Florianópolis), 281-300.
    
    \bibitem{wired}
    Greenberg, A. (n.d.). These Are the Emails Snowden Sent to First Introduce His Epic NSA Leaks. Wired. \url{https://www.wired.com/2014/10/snowdens-first-emails-to-poitras/}


    
    
    \bibitem{chatgpt}
        OpenAI. (2024). \textit{ChatGPT}. OpenAI. Disponível em: \url{https://chatgpt.com/}. Versão 4o mini.
    
   
\end{thebibliography}

\printglossaries

\end{document}
